\documentclass{article}
\usepackage[margin=0.75in]{geometry}
\usepackage{graphicx}
\usepackage{enumitem}
\usepackage{natbib}
\usepackage{multicol}
\usepackage{hyperref}
\hypersetup{
	colorlinks=true,
	linktoc=all,
	linkcolor=blue,
}

\RequirePackage{amsmath, amssymb, amsthm}
\theoremstyle{definition}
\newtheorem{problem}{Problem}

\begin{document}
\title{Homework 2}
\author{{\bf Due:} Feb, 6, midnight}
\date{}
\maketitle

%%%%%%%%%%%%%%%%%%%%%%%%%%%%%%%%%%%%%%%%%%%%%%%%%%%%%%%%%%%%%%
%%%%%%%%%%%%%%%%%%%%%%%%%%%%%%%%%%%%%%%%%%%%%%%%%%%%%%%%%%%%%%
\begin{problem}
Consider the dynamics for the mass-spring system, as depicted in 
Figure~\ref{fig:mass_spring}. 

\vspace{12pt}

\noindent  Here, $k$ is the spring constant and $\rho$ is the friction 
coefficient (yes, the mass $m$ in the figure does not touch the floor, but \textit{assume it does!}).

\vspace{12pt}

\begin{figure}[!ht]
 	\begin{center}
 	\includegraphics[width=0.35\columnwidth]{mass_spring.png}
 	\end{center}
 	\caption{The mass-spring system.}
 	\label{fig:mass_spring}
\end{figure}

\begin{enumerate}[label=(\alph*), noitemsep] 
\item (5 points) Derive the equations of motion for the system, 
assuming the displacement of the mass is \( x(t) \) and the input 
force is \( u(t) \).  

\item (10 points) From the equations of motion, derive the transfer 
function \( H(s) = \frac{X(s)}{U(s)} \), where \( X(s) \) and \( U(s) \) 
are the Laplace transforms of \( x(t) \) and \( u(t) \), respectively.  

\item (10 points) Express the natural frequency \( \omega_n \) and 
the damping ratio \( \zeta \) in terms of \( k \), \( \rho \), and \( m \).  

\end{enumerate}
\end{problem}

%%%%%%%%%%%%%%%%%%%%%%%%%%%%%%%%%%%%%%%%%%%%%%%%%%%%%%%%%%%%%%
\begin{problem}
Consider the transfer function:
\[
H(S) = \dfrac{25}{s^2 + 6 s + 25}
\]
\begin{enumerate}[label=(\alph*), noitemsep]
\item (5 points)
Draw a block diagram for $H(s)$ using integrator, summation, and 
gain blocks.
\item (5 points) Suppose you are given the following time-domain
specs: rise time $t_r \leq 0.6$ and settling time $t_s \leq 1.6$.
(Here we're considering settling time to within 5\% of the 
steady-state value.) Plot the admissible pole locations in the 
$s$-plane corresponding to these two specs. Does this system satisfy 
these specs?
\item (5 points) Repeat the previous problem for the specs: rise 
time $t_r \leq 0.6$, settling time $t_s \leq 1.6$, and magnitude $M_p 
\leq 1/e^2$. Plot the admissible pole locations; does this system 
satisfy these specs?
\item (5 points) Draw a block diagram for $(s+1)H(s)$ using 
integrator, summation, and gain blocks.
\end{enumerate}
\end{problem}
%%%%%%%%%%%%%%%%%%%%%%%%%%%%%%%%%%%%%%%%%%%%%%%%%%%%%%%%%%%%%%
\begin{figure}[!ht]
 	\begin{center}
 	\includegraphics[width=0.5\columnwidth]{unity.png}
 	\end{center}
 	\caption{A diagram of a unity feedback system.}
 	\label{fig:unity}
 \end{figure}
%%%%%%%%%%%%%%%%%%%%%%%%%%%%%%%%%%%%%%%%%%%%%%%%%%%%%%%%%%%%%%
\begin{problem} (15 points) Consider the unity feedback system in
Figure~\ref{fig:unity}. Let the plant's transfer function be given by:
\[
P(s) = \frac{1}{s^3 + 2s^2 + 3s + 1}
\]
Suppose our controller is given by $K(s) = 4$. 
What is the transfer function from $R$ to $Y$? Use the Routh-Hurwitz 
criterion to determine whether this model is stable or not.
\end{problem}
%%%%%%%%%%%%%%%%%%%%%%%%%%%%%%%%%%%%%%%%%%%%%%%%%%%%%%%%%%%%%%
\begin{problem}
(30 points) Consider the six transfer functions given below.  For 
each $G_i(s), i=1,2, \dots 6$, specify the following \textbf{in turn}: 
(a) Poles, (b) Zeros (if any), (c) Stable or unstable, and (d) 
Steady-state gain \textbf{before proceeding to the next}. Use these
answers to match each of the six transfer functions with one of
the unit step responses in the figure below. All responses were
generated with zero initial conditions.
\begin{table}[h!]
\begin{center}
\begin{tabular}{llll}
$G_1(s) = \dfrac{-4s+4}{s^2 +3 s + 4}$
&
$G_2(s) = \dfrac{4}{s^2 + 0.3 s + 4}$
&
$G_3(s) = \dfrac{4}{s^2 + 3 s + 4}$
\\ 
&&& 
\\
$G_4(s) = \dfrac{-s+3}{s-1}$
&
$G_5(s) = \dfrac{300}{s^2+101s+100}$
&
$G_6(s) = \dfrac{-3}{s+1}$
\end{tabular}
\end{center}
\end{table}

\vspace{-0.3in}
\begin{figure}[h]
  \begin{center}
  \includegraphics[width=7.5in]{ExtraPolesAndZeros.png}
  \end{center}
\end{figure}
\end{problem}
\vspace{-0.3in}

%%%%%%%%%%%%%%%%%%%%%%%%%%%%%%%%%%%%%%%%%%%%%%%%%%%%%%%%%%%%%%}
\begin{problem}
 (15 points) Without a computer, determine whether or not the following 
 polynomials have any RHP roots:
\begin{enumerate}[label=(\alph*), noitemsep]
\item 
$s^6 + 2s^5 +3s^4 + s^3 + s^2 - 3s + 5$
\item
$s^4 + 10s^3 + 10s^2 + 20s + 1$
\item
$s^4 + 10s^3 + 10s^2  + 1$
\end{enumerate}
\end{problem}
\end{document}